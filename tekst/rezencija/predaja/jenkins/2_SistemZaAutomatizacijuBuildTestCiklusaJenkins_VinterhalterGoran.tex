% !TEX encoding = UTF-8 Unicode

\documentclass[a4paper]{article}

\usepackage[T2A]{fontenc} % enable Cyrillic fonts
\usepackage[utf8x,utf8]{inputenc} % make weird characters work
\usepackage[serbian]{babel}
%\usepackage[english,serbianc]{babel}
\usepackage{amssymb}

\usepackage{color}
\usepackage{url}
\usepackage[unicode]{hyperref}
\hypersetup{colorlinks,citecolor=green,filecolor=green,linkcolor=blue,urlcolor=blue}

\newcommand{\odgovor}[1]{\textcolor{blue}{#1}}

\begin{document}

\title{Jenkins ukratko\\
  \small{Uroš Milenković, Nikola Sojčić, Bojan Nestorović
}}

%%%%%%%%%%%%%%%%%%%%%%%%%%%%%%%%%%%%%%%%%%%%%%%%%%%%%%%%%%%%%%%%%%%%%%%%%%%%%%%%%%%%%%%%%%
\author{Recenzija: Goran Vinterhalter}
%ime autora recenzije neće biti predato autorima seminarskog rada
%%%%%%%%%%%%%%%%%%%%%%%%%%%%%%%%%%%%%%%%%%%%%%%%%%%%%%%%%%%%%%%%%%%%%%%%%%%%%%%%%%%%%%%%%%



\maketitle

\section{O čemu rad govori?}
Rad opisuje osnovne mogućnosti softverskog paketa Jenkins koji automatizuje
procese: prevođenja, testiranja, arhiviranja i deployment-a fokusirajući se na 
sekvencionalni metod izvršavanja.

\section{Krupne primedbe i sugestije}

Rad potpuno ignoriše \textit{"eng. pipeline"} pristup i fokusira se isključivo na slobodne
projekte koji dozvoljavaju samo sekvencijalno izvršavanje zadataka. Tvrđenje da slobodni
projekat pruža maksimalnu fleksibilnost s'toga nije tačna.

Opis opcije parametrizovane izgradnje (\textit{eng. parametrized build}) treba
dopuniti konkretnijim primerima. Neupućenom čitaocu je teško razumeti kakve kakve su
to opcije koje se prosleđuju i koji format bi bio pogodniji za koju situaciju.

\section{Sitne primedbe}
Rad se u nekim trenucima čini kao kratak tutorijal za početnike umesto uopštenog
pregleda mogućnosti. Navođenje detalja kao što su opisi instalacije proširenja
(git ...) ili specifičnosti kreiranja novog projekta nisu toliko bitne za 
samu temu.

Trebalo bi izbegavati rečenice kao što su:
\begin{itemize}
  \item "pokušaćemo da zagrebemo površinu ovog složenog sistema", jer rad čini manje vrednim.
\end{itemize}
Postoji nekoliko sitnih grešaka pri kucanju.

  


\section{Provera sadržajnosti i forme seminarskog rada}

\begin{enumerate}
\item Da li rad dobro odgovara na zadatu temu?\\
  Da, rad je za temu imao da opiše mogućnosti i prednosti softvera Jenkins što je i opisano.
\item Da li je nešto važno propušteno?\\
  Pipeline (ili workflow ) je postao standard za upravljanje kompleksnijim nelinearnim
  izvršavanjem. U ovom radu pipeline je potpuno zanemaren. Takođe postoji specijalna
  terminologija koja se koristi za definisanje jednog pipeline konfiguracionog fajla koju bi trebalo
  navesti i predstaviti jedan šematski prikaz.
\item Da li ima suštinskih grešaka i propusta?\\
  Nema.
\item Da li je naslov rada dobro izabran?\\
  Da.
\item Da li sažetak sadrži prave podatke o radu?\\
  Da.
\item Da li je rad lak-težak za čitanje?\\
  Rad je pitak i lepo ilustrovan uz par primedbi navedenih gore.
\item Da li je za razumevanje teksta potrebno predznanje i u kolikoj meri?\\
  Jeste. \\
  Neophodno je razumevanje oblasti:
  \begin{enumerate}
      \item Životni ciklus razvoja softvera.
      \item Version sistemi, prvenstveno git.
      \item Unit testovi.
  \end{enumerate}
\item Da li je u radu navedena odgovarajuća literatura?\\
  Da.
\item Da li su u radu reference korektno navedene?\\
  U radu postoji samo jedna referenca. Potrebno je da svako veće tvrđenje bude podržano
  sa odgovarajućom referencom. Recimo navodi se da je najznačajnija stvar za jenkins integracija
  sa sistemom za kontrolu verzija. Zašto to tvrđenje nema referencu a tvrđenje za Unit testove ima?
\item Da li je struktura rada adekvatna?\\
  Struktura je adekvatna, ali fale reference.
\item Da li rad sadrži sve elemente propisane uslovom seminarskog rada (slike, tabele, broj strana...)?\\
  Da.
\item Da li su slike i tabele funkcionalne i adekvatne?\\
  Da, jedino bi valjalo da se pojavljuju malo ranije u tekstu radi lakšeg praćenja.
  Poslednju tabelu bi trebalo Transponovati. Praksa je da se tabele koje predstavljaju
  razlike između 2 i više softvera prave tako da jedna opcija odgovara jednom redu.
\end{enumerate}



\section{Ocenite sebe}
% Napišite koliko ste upućeni u oblast koju recenzirate: 
% a) ekspert u datoj oblasti
% b) veoma upućeni u oblast
% c) srednje upućeni
d) malo upućeni \\
% e) skoro neupućeni
% f) potpuno neupućeni
% Obrazložite svoju odluku
Tokom obrazovanja nisam imao kontakt sa rešenjima kao što su Jenkins.
Međutim zbog kurseva kao što su: Razvoj softvera 1, 2 i Microsoft kurseva imam 
dobru ideju o potrebama razvojnih timova.

%%%%%%%%%%%%%%%%%%%%%%%%%%%%%%%%%%%%%%%%%%%%%%%%%%%%%%%%%%%%%%%%%%%%%%%%%%%%%%%%%%%%%%%%%%
\section{Poverljivi komentari}
% Poverljivi komentari neće biti prosleđeni autorima seminarskog rada.
% Ukoliko nemate poverljivih komentara, ovaj deo može da ostane prazan.
%%%%%%%%%%%%%%%%%%%%%%%%%%%%%%%%%%%%%%%%%%%%%%%%%%%%%%%%%%%%%%%%%%%%%%%%%%%%%%%%%%%%%%%%%%


\end{document}


