% !TEX encoding = UTF-8 Unicode

\documentclass[a4paper]{article}

\usepackage[T2A]{fontenc} % enable Cyrillic fonts
\usepackage[utf8x,utf8]{inputenc} % make weird characters work
\usepackage[serbian]{babel}
%\usepackage[english,serbianc]{babel}
\usepackage{amssymb}

\usepackage{color}
\usepackage{url}
\usepackage[unicode]{hyperref}
\hypersetup{colorlinks,citecolor=green,filecolor=green,linkcolor=blue,urlcolor=blue}

\newcommand{\odgovor}[1]{\textcolor{blue}{#1}}

\begin{document}

\title{
  Opis procesa analize i otklanjanja grešaka
  programske podrške primenom alata Velgrind
\\ \small{Dragutin Ilić, Karadžić Aleksandra, šuka Aleksandar }
}

%%%%%%%%%%%%%%%%%%%%%%%%%%%%%%%%%%%%%%%%%%%%%%%%%%%%%%%%%%%%%%%%%%%%%%%%%%%%%%%%%%%%%%%%%%
\author{Recenzija: Goran Vinterhalter}
%ime autora recenzije neće biti predato autorima seminarskog rada
%%%%%%%%%%%%%%%%%%%%%%%%%%%%%%%%%%%%%%%%%%%%%%%%%%%%%%%%%%%%%%%%%%%%%%%%%%%%%%%%%%%%%%%%%%


\maketitle

%Recenziju predajete u tex obliku, budite uredni!

Pod komentarima su data objašnjenja za svaku navedenu stavku.

\section{O čemu rad govori?}
Rad opisuje mogućnosti i korišćenje alata Valgrind, prvenstveno 6 osnovnih alata za
testiranje propusta u oblastima provere glavne memorije, keš iskorišćenosti
i višenitnog programiranja.

\section{Krupne primedbe i sugestije}
Rad propušta da pomene i ostale alatae: DHAT, SGcheck, BBV. Iako su poslednja
dva eksperimentalna trebalo bi ih navesti zbog kompletnosti.\\
Ime softvera bi trebalo ostaviti u originalu Valgrind. Po Nordiskoj mitologiji
Valgrind su glavna vrata ulaska u Valhalu i izgovaraju se sa 'a'.

\section{Sitne primedbe}
Pored termina "mrtva petlja" trebalo bi zbog potpunosti navesti \textit{(eng. deadloack)}
Rad suviše pažnje posvećuje alatima za višenitno programiranje.

\section{Provera sadržajnosti i forme seminarskog rada}

\begin{enumerate}
\item Da li rad dobro odgovara na zadatu temu?\\
  Da.
\item Da li je nešto važno propušteno?\\
  Treba pomenuti i preostala tri alata.
\item Da li ima suštinskih grešaka i propusta?\\
  Ne.
\item Da li je naslov rada dobro izabran?\\
  Da.
\item Da li sažetak sadrži prave podatke o radu?\\
  Da.
\item Da li je rad lak-težak za čitanje?\\
  Rad je povremeno suviše detaljan.
\item Da li je za razumevanje teksta potrebno predznanje i u kolikoj meri?\\
  Potrebno je osnovno poznavanje: programiranja, arhitekture i organizacije računara kao i problema višenitnog programiranja.
\item Da li je u radu navedena odgovarajuća literatura?\\
  Da.
\item Da li su u radu reference korektno navedene?\\
  Da.
\item Da li je struktura rada adekvatna?\\
  Da.
\item Da li rad sadrži sve elemente propisane uslovom seminarskog rada (slike, tabele, broj strana...)?\\
  Da.
\item Da li su slike i tabele funkcionalne i adekvatne?\\
  Slike su veoma loše rezolucije. Autor se savetuje da sve slike sa plavom
  pozadinom zameni sa tekstom.
\end{enumerate}

\section{Ocenite sebe}
% Napišite koliko ste upućeni u oblast koju recenzirate: 
% a) ekspert u datoj oblasti
% b) veoma upućeni u oblast
% c) srednje upućeni
% d) malo upućeni 
e) skoro neupućeni
% f) potpuno neupućeni
% Obrazložite svoju odluku
Nisam se nikad susreo sa alatima ovog tipa.
Poznata mi je problematika kojom se bavi ali ništa dalje.

%%%%%%%%%%%%%%%%%%%%%%%%%%%%%%%%%%%%%%%%%%%%%%%%%%%%%%%%%%%%%%%%%%%%%%%%%%%%%%%%%%%%%%%%%%
\section{Poverljivi komentari}
% Poverljivi komentari neće biti prosleđeni autorima seminarskog rada.
% Ukoliko nemate poverljivih komentara, ovaj deo može da ostane prazan.
%%%%%%%%%%%%%%%%%%%%%%%%%%%%%%%%%%%%%%%%%%%%%%%%%%%%%%%%%%%%%%%%%%%%%%%%%%%%%%%%%%%%%%%%%%


\end{document}


